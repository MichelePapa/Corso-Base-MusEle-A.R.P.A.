\documentclass[15pt,a4paper]{report} 

%impostazioni di pagina
\usepackage{fancyhdr}

%FONT
\usepackage{mathpazo} % math & rm
\linespread{.75}        % Palatino needs more leading (space between lines)
\usepackage[scaled]{helvet} % ss
\usepackage{courier} % tt
\normalfont
\usepackage[T1]{fontenc}
\usepackage[italian]{babel}
\usepackage[latin1]{inputenc}

% resize title

\usepackage{titlesec}
\titleformat{\chapter}[display]
  {\normalfont\sffamily\huge\bfseries}%\color{black}}
  {\chaptertitlename\ \thechapter}{5pt}{\large}
\titleformat{\section}
  {\normalfont\sffamily\normalsize\bfseries}%\color{grey}}
  {\thesection}{1em}{}
  
 %colore

\usepackage{xcolor}
  \usepackage {textcomp}

%titolatura

\usepackage{titling}
\newcommand{\subtitle}[1]{%
  \posttitle{%
    \par\end{center}
    \begin{center}\large#1\end{center}
    \vskip0.3em}%
}
\usepackage{graphicx}
\usepackage{geometry}
\geometry{a4paper,top=3cm,bottom=3cm,left=3.5cm,right=3.5cm,%
heightrounded,bindingoffset=5mm}
\title{\textbf{MUSICA ELETTRONICA}}
\subtitle{\textsc{Corso Base}}
\author{DOCENTE \\ \\ Michele Papa}
\date{2018}
\pagenumbering{arabic}

%apertura documento
\begin{document}
\maketitle

\chapter*{Lezione I - 3 Marzo 2018}



Il corso base in Musica Elettronica serve a formare un gusto musicale verso il mondo dell'elettronica e dell'elettroacustica, cercando di adempiere alle competenze base che un corso creativo dovrebbe dare:
\begin{enumerate}
\item{Il campionamento}
\item{Trasformazione di un suono}
\item{Competenze compositive di base}
\item{Necessit� storico-culturali di un elettronica}
\item{Conoscenza delle forme musicali}
\item{Sviluppo della propria creativit�}
\end{enumerate}




\chapter*{Sabato 3 Marzo \\ Lezione I}
\addcontentsline{toc}{chapter}{Lezione I - 180303 - Campionamento, armonia e cadenze}

\textit{CAMPIONAMENTO} \\
Il suono � alla base di tutti i nostri rapporti umani e soprattutto � ci� che ha reso possibile la comunicazione e il riconoscimento di esseri viventi dello stesso genere anche divisi da distanze enormi (per esempio l'utilizzo dello yodel per comunicare). In questa prima lezione si parler� del suono e le sue caratteristiche formanti, percettive e fisiche. Il campionamento, la tipologia di software da utilizzare.\\
Legato al campionamento verr� studiato lo spettro e le formanti dei suoni da creare. Il primo passo sembra arduo, ma via via che si capiscono le cose, si diventer� subito consci di quello che si andr� a fare sui computer. \\
\\
\textit{COMPOSIZIONE}\\ 
Movimento delle voce all'interno di una struttura armonica. Spiegazione di cadenze. Triadi. Movimenti del basso.
\\
\chapter*{Sabato 10 Marzo \\ Lezione II}
\addcontentsline{toc}{chapter}{Lezione II - 180310 - Introduzione Max, Ableton - Triadi}

\textit{MAXMSP} \\
Lezione introduttiva. Cycle, forme d'onda e utilizzo di un sistema a blocchi. \\ \\
\textit{ABLETON LIVE} \\
Lezione Introduttiva. \\
\\
\textit{COMPOSIZIONE} \\
\begin{enumerate}
\item{Triade diminuite e aumentate}
\item{Studio storico dei cambiamenti accordali legati a stili musicali}
\end{enumerate}

\chapter*{Sabato 17 Marzo \\ Lezione III}
\addcontentsline{toc}{chapter}{Lezione III - 180317 - Sintetizzatori e registrazione - Ritmi e percussioni}

\textit{MAXMSP} \\
Inizio del lavoro di costruzione di un sintetizzatore. \\ \\
\textit{ABLETON LIVE} \\
Inizio studio di registrazioni di Loop e sviluppo di una forma musicale. \\
\\
\textit{COMPOSIZIONE} \\
Ritmi e percussioni.

\chapter*{Sabato 24 Marzo \\ Lezione IV}
\addcontentsline{toc}{chapter}{Lezione IV - 180324 - Elaborazione audio. Forma canzone}

\textit{MAXMSP} \\
Utilizzo degli stessi comandi che poi serviranno per esportare i vostri sintetizzatori ed effetti ed utilizzarli tramite maxForLive su Ableton Live. \\ \\
\textit{ABLETON LIVE} \\
Utilizzo di un Sequencer e sviluppo di una mente critica sull'editing audio. Filtri, compressori, effetti.\\
\\
\textit{COMPOSIZIONE} \\
La forma canzone.

\chapter*{Sabato 7 Aprile \\ Lezione V}
\addcontentsline{toc}{chapter}{Lezione V - 180407 - Melodia e armonia}

\textit{MAXMSP} \\
Costruire una melodia con un sintetizzatore. \\ \\
\textit{ABLETON LIVE} \\
Costruire una melodia e un armonia.\\
\\
\textit{COMPOSIZIONE} \\
Differenze tra melodia e armonia. Movimento verticale e orizzontale delle voci. 

\chapter*{Sabato 14 Aprile \\ Lezione VI}
\addcontentsline{toc}{chapter}{Lezione VI - 180414 - Punto contro punto}

\textit{MAXMSP} \\
Elaborazione digitale \\ \\
\textit{ABLETON LIVE} \\
Elaborazione in tempo "reale" - i controller MIDI.\\
\\
\textit{COMPOSIZIONE} \\
Indipendenza delle voci. Il contrappunto.

\chapter*{Sabato 21 Aprile \\ Lezione VII}
\addcontentsline{toc}{chapter}{Lezione VII - 180421 - Laboratorio creativo I e laboratorio di saldatura}

\begin{enumerate}
\item{Laboratorio creativo I} 
\item{Laboratorio di saldatura} 
\end{enumerate}

Si cercher� di mettere in azione tutte le competenze di base acquisite. \\ \\

\chapter*{Sabato 28 Aprile \\ Lezione VIII}
\addcontentsline{toc}{chapter}{Lezione VIII - 180428 - Laboratorio II (Eventuale Live)}

Laboratorio creativo II. \\
\\
Si parler� delle scelte fatte e dei risultati raggiunti. Suoniamo.

\end{document}




